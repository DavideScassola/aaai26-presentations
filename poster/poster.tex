% !TeX program=lualatex
\documentclass[25pt, a0paper, portrait]{conferenceposter}

%\geometry{paperheight=48in,paperwidth=36in} % for different sizes than A paper

\usepackage[english]{babel}
\usepackage{microtype}
\usepackage{enumitem}

%\usepackage[T1]{fontenc}
%\usepackage[defaultsans,default]{opensans}
\usepackage{fontenc} % use with lualatex or xelatex
\usepackage{arev} % math font
%\setmainfont{Open Sans}


\usepackage{blindtext}



\definecolorpalette{myColorPalette}{
	\definecolor{colorOne}{HTML}{ECF1FC} % Main background color that is used as the background color of the poster.
	\definecolor{colorTwo}{HTML}{ECF1FC} % Note color that is used for the frame and in a lighter shade as the background color of notes.
	\definecolor{colorThree}{HTML}{1A3A52} % Accent color that is used as the background of the poster title and the blocks' titles.
	\definecolor{bgcolorAlt}{HTML}{FCFCFF} % Second/alternative background color that is used as the background color of the blocks, as the foreground color in the titles and as the background color for the head of the poster.
	\definecolor{fgcolor}{HTML}{222244} % Foreground color that is used as the main text color inside the blocks.
}
\usecolorstyle[colorPalette=myColorPalette]{conferenceColorStyle}


% --------------------------------------
% ## Logo
% If you want to show the logo of your institution/university on the top, you
% simple need to adjust the `\titlegraphic` command like
% ```latex
% \titlegraphic{\includegraphics[height=.08\textheight]{logo.png}}
% ```
% where `logo.png` refers to the image file of the logo.


% When you want to show multiple logos, e.g., because it is a collaboration, you
% can do the following.
% ```latex
% \titlegraphic{\parbox{\titlewidth}{\includegraphics[height=.06\paperheight]{logo1.png} \hfill \includegraphics[height=.06\paperheight]{logo2.png} \hfil \includegraphics[height=.06\paperheight]{logo3.png}}}
% ```
% --------------------------------------


\newcommand{\titlespace}{\hspace{30pt}} % adjust gap if needed

\title{\centering\boldmath{}Graph-Conditional\titlespace Flow\titlespace Matching\titlespace \\ for\titlespace Relational\titlespace Data\titlespace Generation}
\author{\centering Davide Scassola\affilmark{1,2} \hfil Sebastiano Saccani\affilmark{2}  \hfil Luca Bortolussi\affilmark{1}}
\institute{\centering %
\affilmark{1} AI \textsc{Lab}, University of Trieste, Trieste, Italy \hspace{50pt}
\affilmark{2} Aindo SpA, AREA Science Park, Trieste, Italy\\
}

%\titlegraphic{\includegraphics[height=.08\paperheight]{../assets/logos/logo_ailab.pdf}}

\titlegraphic{\centering\parbox{\titlewidth}{\includegraphics[height=.06\paperheight]{../assets/logos/logo_UNITS.png}
\hfill
\includegraphics[height=.06\paperheight]{../assets/logos/logo_ailab.pdf}
\hfill
\includegraphics[height=.03\paperheight]{../assets/logos/logo_aindo.png}}}


\usetitlestyle[titlebackgroundheight=.09\paperheight,titlegraphictotitledistance=1cm]{conferenceTitleStyle}


\begin{document}
\maketitle







% -----------------------------------------------------------------------------------------------------
% \block{SUUUUUUUUUUUUUUUUUUUUUUUUUUUUUUUUUUUUM}
% {
%     text text 
% }
% -----------------------------------------------------------------------------------------------------

\begin{columns}
\column{0.5}

% -----------------------------------------------------------------------------------------------------
\block{1: The Problem}
{
Data is precious but often sensitive. How can we share useful information while preserving privacy?
Synthetic data is the solution: learn a generative model on real data and share new samples.
Synthetic data is as useful as the original but does not expose sensitive information.

\begin{center}
	\includegraphics[width=0.45\textwidth]{../assets/images/synthetic_data_infographic.pdf}
\end{center}
}
% -----------------------------------------------------------------------------------------------------
% -----------------------------------------------------------------------------------------------------
\block{2: Generating Relational Data is Difficult}{
  \begin{minipage}[t]{0.48\linewidth}
    \vspace{0pt}
    %\setlength{\parskip}{12pt}\setlength{\parindent}{0pt}
    Relational data = multiple tables (structured data) + links between records.

    \vspace{22pt}
    
    Current methods fail at:
    \begin{itemize}
      \item Modelling relationships between any record indirectly connected
      \item Adapting to complex foreign key structures
    \end{itemize}
  \end{minipage}\hfill
  \begin{minipage}[t]{0.48\linewidth}
    \vspace{0pt}
    \centering
    \includegraphics[width=0.9\textwidth,page=1]{../assets/images/graphics.pdf}
  \end{minipage}
}
% -----------------------------------------------------------------------------------------------------
% -----------------------------------------------------------------------------------------------------
\block{3: Relational Data is a Graph}{
    \begin{itemize}
        \item Records = nodes
        \item Foreign key references = edges
        \item i.i.d. samples = connected components
    \end{itemize}

    \centering
    \includegraphics[width=0.45\textwidth,page=2]{../assets/images/graphics_compressed.pdf}

    }
% -----------------------------------------------------------------------------------------------------



\column{0.5}
% -----------------------------------------------------------------------------------------------------
\block{4: Graph-Conditional generation}{
    \begin{itemize}
        \item Sample the graph structure with a scalable method
        \item Use flow matching (deep generative modelling) to "fill" the empty graph with the features
        \item Use a GNN for flow matching
    \end{itemize}

    {\centering
    \includegraphics[width=0.45\textwidth,page=3]{../assets/images/graphics_compressed.pdf}}

    Advantages:
    \begin{itemize}
        \item \textbf{Maximally expressive}: any record in the same connected component is modelled jointly
        \item \textbf{Flexible}: any foreign key graph can be used
        \item \textbf{Scalable}: no need for instantiating (or generating) the edge matrix
    \end{itemize}
    }
% -----------------------------------------------------------------------------------------------------
% -----------------------------------------------------------------------------------------------------
\block{Flow Matching}
{
\textbf{Generation}: solving an ODE that maps noise into data.

\textbf{Training}: learn the vector field guiding the ODE by denoising samples with added noise (same as diffusion).

With variational FM a denoising distribution is learned: $p()$, that can be fully factorized without loss of generality.

{\centering
\includegraphics[width=0.4\textwidth,page=1]{../assets/images/flow_matching.pdf}
}
}
% -----------------------------------------------------------------------------------------------------






%\note[targetoffsetx=-9cm, targetoffsety=-5.5cm, width=0.5\linewidth]{This is a note...}
\end{columns}

\begin{columns}
\column{0.5}
\block{A figure}
{
\begin{center}
	\includegraphics[width=0.2\textheight]{../assets/images/synthetic_data_infographic.pdf}
\end{center}
}

\block{Outer Block}{
\blindtext
\innerblock{Inner Block}{\blindtext}
}


\column{0.5}
\block{Description of the figure}{\Large\blindtext[2]}





















\end{columns}
\end{document}