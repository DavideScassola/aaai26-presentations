\pdfobjcompresslevel=0
\documentclass[25pt, a0paper, portrait]{conferenceposter}

%\geometry{paperheight=48in,paperwidth=36in} % for different sizes than A paper

\usepackage[english]{babel}
\usepackage{microtype}
\usepackage{enumitem}
\usepackage{amsmath}
\usepackage{booktabs}
\usepackage{natbib}
\bibliographystyle{plainnat}
\setlength{\bibsep}{2pt}
\makeatletter
\renewcommand\@biblabel[1]{\hfill #1.}
\makeatother

\usepackage{fontenc} % correct encoding for Arev
\usepackage{arev} % math font
%\setmainfont{Open Sans}


\usepackage{blindtext}



\definecolorpalette{myColorPalette}{
	\definecolor{colorOne}{HTML}{ECF1FC} % Main background color that is used as the background color of the poster.
	\definecolor{colorTwo}{HTML}{ECF1FC} % Note color that is used for the frame and in a lighter shade as the background color of notes.
	\definecolor{colorThree}{HTML}{1A3A52} % Accent color that is used as the background of the poster title and the blocks' titles.
	\definecolor{bgcolorAlt}{HTML}{FCFCFF} % Second/alternative background color that is used as the background color of the blocks, as the foreground color in the titles and as the background color for the head of the poster.
	\definecolor{fgcolor}{HTML}{222244} % Foreground color that is used as the main text color inside the blocks.
}
\usecolorstyle[colorPalette=myColorPalette]{conferenceColorStyle}


% --------------------------------------
% ## Logo
% If you want to show the logo of your institution/university on the top, you
% simple need to adjust the `\titlegraphic` command like
% ```latex
% \titlegraphic{\includegraphics[height=.08\textheight]{logo.png}}
% ```
% where `logo.png` refers to the image file of the logo.


% When you want to show multiple logos, e.g., because it is a collaboration, you
% can do the following.
% ```latex
% \titlegraphic{\parbox{\titlewidth}{\includegraphics[height=.06\paperheight]{logo1.png} \hfill \includegraphics[height=.06\paperheight]{logo2.png} \hfil \includegraphics[height=.06\paperheight]{logo3.png}}}
% ```
% --------------------------------------


\newcommand{\titlespace}{\hspace{30pt}} % adjust gap if needed

\title{\centering\boldmath{}Graph-Conditional\titlespace Flow\titlespace Matching\titlespace \\ for\titlespace Relational\titlespace Data\titlespace Generation}
\author{\centering Davide Scassola\affilmark{1,2} \hfil Sebastiano Saccani\affilmark{2}  \hfil Luca Bortolussi\affilmark{1}}
\institute{\centering %
\affilmark{1} AI \textsc{Lab}, University of Trieste, Trieste, Italy \hspace{50pt}
\affilmark{2} Aindo SpA, AREA Science Park, Trieste, Italy\\
}

%\titlegraphic{\includegraphics[height=.08\paperheight]{../assets/logos/logo_ailab.pdf}}

\titlegraphic{%
  \centering
  \parbox{\titlewidth}{%
    \begin{minipage}[c][.06\paperheight][c]{\titlewidth}
      \centering
      \raisebox{-0.5\height}{\includegraphics[height=.05\paperheight]{../assets/logos/logo_UNITS.png}}%
      \hspace*{0.05\titlewidth}%
      \raisebox{-0.5\height}{\includegraphics[height=.05\paperheight]{../assets/logos/logo_ailab.pdf}}%
      \hspace*{0.05\titlewidth}%
      \raisebox{-0.5\height}{\includegraphics[height=.035\paperheight]{../assets/logos/logo_aindo.png}}%
    \end{minipage}%
  }%
}


\usetitlestyle[titlebackgroundheight=.09\paperheight,titlegraphictotitledistance=0mm]{conferenceTitleStyle}


\begin{document}
\fontsize{28pt}{33pt}\selectfont
\maketitle







% -----------------------------------------------------------------------------------------------------
% \block{SUUUUUUUUUUUUUUUUUUUUUUUUUUUUUUUUUUUUM}
% {
%     text text 
% }
% -----------------------------------------------------------------------------------------------------

\begin{columns}
\column{0.5}

% -----------------------------------------------------------------------------------------------------
\block{1: The Problem}
{
\textbf{Data is valuable but often sensitive}

Synthetic data generated by learning a generative model is the solution.
It is as useful as the original but does not expose sensitive information.

% How can we share useful information while preserving privacy?
% Synthetic data is the solution: learn a generative model on real data and share new samples.
% Synthetic data is as useful as the original but does not expose sensitive information.

\begin{center}
	\includegraphics[width=0.4\textwidth]{../assets/images/synthetic_data_infographic.pdf}
\end{center}
%}
% -----------------------------------------------------------------------------------------------------
% -----------------------------------------------------------------------------------------------------
%\block{2: Generating Relational Data is Difficult}{
%\subsection*{But generating relational data is difficult}
%\textbf{But generating relational data is difficult}

  \begin{minipage}[t]{0.48\linewidth}
    \textbf{But generating relational data is difficult}

    \vspace{0pt}
    
    %\setlength{\parskip}{12pt}\setlength{\parindent}{0pt}
    Relational data = multiple tables (structured data) + links between records.

    \vspace{22pt}
    
    Current methods fail at:
    \begin{itemize}
      \item Modelling relationships between any record indirectly connected
      \item Adapting to complex foreign key structures
    \end{itemize}
  \end{minipage}\hfill
  \begin{minipage}[t]{0.48\linewidth}
    \vspace{0pt}
    \centering
    \includegraphics[width=0.7\textwidth,page=1]{../assets/images/schema_bigger.pdf}
  \end{minipage}
}
% -----------------------------------------------------------------------------------------------------
% -----------------------------------------------------------------------------------------------------
\block{2: Graph-Conditional Generation}{
    \textbf{Relational data is a graph}:
    \begin{itemize}
        \item Records = nodes, foreign key references = edges
        \item Connected components are the i.i.d. samples 
    \end{itemize}

    \begin{center}
    \includegraphics[width=0.45\textwidth,page=2]{../assets/images/graphics_compressed.pdf}
    \end{center}

    \bigskip
    \textbf{Our approach}:
    \begin{itemize}
        \item Sample the graph structure $G$ with a scalable method
        \item Use flow matching to "fill" the empty graph with content $X$
        %\item Use a GNN for flow matching
    \end{itemize}

    \begin{center}
    \includegraphics[width=0.4\textwidth,page=3]{../assets/images/graphics_compressed.pdf}
    \end{center}

    \begin{itemize}
        \item \textbf{Expressive}: records in a connected component are modelled jointly
        \item \textbf{Flexible}: any foreign key graph can be used
        \item \textbf{Scalable}: flow matching scales to large dimensionalities + no need for materializing (or generating) the dense adjacency matrix
    \end{itemize}
    }
% -----------------------------------------------------------------------------------------------------
\column{0.5}

% -----------------------------------------------------------------------------------------------------
\block{3: Flow Matching for Relational Data}
{
Flow Matching (FM) \citep{flow_matching} is a diffusion-like technique to learn ODEs that transform a simple distribution into a complex one.

\bigskip

\textbf{Training}: learn a denoiser by denoising samples with added noise.
\begin{center}
\includegraphics[width=0.45\textwidth,page=1]{../assets/images/training.pdf}
\end{center}
\textbf{Generation}: start from noise and solve the ODE using the learnt vector field: $v_t^\theta(X_t) = \mathbb{E}_{\mathbf{X}_1 \sim \textcolor{black}{p_\theta(X_1 \mid  X_t, G)}} \left[ \textcolor{black}{u_t(X_t \mid X_1)} \right] $ where $u_t$ depends on $p_t(X_t \mid X_1)$.
\begin{center}
\includegraphics[width=0.45\textwidth,page=1]{../assets/images/sampling.pdf}
\end{center}

}
% -----------------------------------------------------------------------------------------------------

% -----------------------------------------------------------------------------------------------------
\block{4: GNN-Based Denoiser}
{
With variational FM \citep{eijkelboom2024variational} a denoising distribution $p_\theta$ is learnt, which expected value we compute in the following way:
\begin{itemize}
  \item A GNN computes node embeddings for each record
  \begin{equation*}
    \mathbf{\varepsilon}_t^i = \text{GNN}_\theta\left(X_t, G, t \right)^i
  \end{equation*}

  \item MLPs exploit node embeddings (graph context) to denoise records
  \begin{equation*}
    \hat{x}_1^i = \text{MLP}_\theta\left(x_t^i, \mathbf{\varepsilon}_t^i, t \right)
  \end{equation*}
\end{itemize}

\begin{center}
    \includegraphics[width=0.45\textwidth, page=1]{../assets/images/denoiser_shorter.pdf}
\end{center}
}
% -----------------------------------------------------------------------------------------------------

% -----------------------------------------------------------------------------------------------------
\block{5: Results}
{
  \begin{itemize}
    \item Fidelity is measured as the accuracy of an XGBoost classifier, trained to distinguish real from generated rows (the lower the better)
    \item We achieve state-of-the-art performances in terms of fidelity
    \item No privacy leaks from DCR analysis
  \end{itemize}

\centering
\small
\resizebox{0.45\textwidth}{!}{
\begin{tabular}{lcccccc}
\toprule
 & \textbf{AirBnB} & \textbf{Biod.} & \textbf{CORA} & \textbf{IMDB} & \textbf{Rossmann} & \textbf{Walmart}\\
\midrule
Ours & $\mathbf{0.58 \pm 0.03}$ & $\mathbf{0.59 \pm 0.02}$ & $0.63 \pm 0.02$ & $\mathbf{0.59 \pm 0.03}$ & $\mathbf{0.51 \pm 0.01}$ & $\mathbf{0.73 \pm 0.01}$ \\
Ours (no GNN) & $0.70 \pm 0.005$ & $0.86 \pm 0.004$ & $0.62 \pm 0.004$ & $0.89 \pm 0.002$ & $0.75 \pm 0.01$ & $0.91 \pm 0.04$ \\
\cite{hudovernik2024relational} & $0.67 \pm 0.003$ & $0.83 \pm 0.01$ & $\mathbf{0.60 \pm 0.01}$ & $0.64 \pm 0.01$ & $0.77 \pm 0.01$ & $0.79 \pm 0.04$ \\
ClavaDDPM & $\approx 1$ & - & - & $0.83 \pm 0.004$ & $0.86 \pm 0.01$ & $0.74 \pm 0.05$ \\
RCTGAN  & $0.98 \pm 0.001$ & $0.88 \pm 0.01$ & $0.73 \pm 0.01$ & $0.95 \pm 0.002$ & $0.88 \pm 0.01$ & $0.96 \pm 0.02$ \\
REaLTabF. & $\approx 1$ & - & - & - & $0.92 \pm 0.01$ & $\approx 1$ \\
SDV & $\approx 1$ & $0.98 \pm 0.01$ & $\approx 1$ & - & $0.98 \pm 0.003$ & $0.90 \pm 0.03$ \\
\bottomrule
\end{tabular}
}
}
% -----------------------------------------------------------------------------------------------------

%\note[targetoffsetx=-9cm, targetoffsety=-5.5cm, width=0.5\linewidth]{This is a note...}
\end{columns}


\begin{columns}
\column{0.8}
\block{}
{
\renewcommand{\section}[2]{} % Suppress the "References" heading from \bibliography

%\vspace{-0.8em}
\textbf{References}
\bibliography{poster}
}
\column{0.2}
\block{}
{ 
    \centering
    \includegraphics[width=0.11\textwidth]{../assets/images/qr_code_arxiv.pdf}
}
\end{columns}

% \begin{columns}
% \column{0.5}

% \block{}
% {
% \small
% \bibliography{poster}
% }

% \column{0.5}

% \block{}
% {
% {\centering
%     \includegraphics[width=0.45\textwidth, page=5]{../assets/images/graphics_compressed.pdf}
% }
% }


\end{document}